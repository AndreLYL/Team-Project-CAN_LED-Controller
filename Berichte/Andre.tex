\begin{document}
\chapter{Hardware}
\label{cha:Hardware}

1.Bestandteil der Hardware

Die Hardwareschaltung dieses Projekts besteht aus vier Teilen: Mikrocontroller, CAN Transceiver, Module der Spannungsversorgung und LED-Leiste. Das Leistungsmodul ist dafür verantwortlich, das gesamte Hardware-Schaltermodul mit der entsprechenden Spannung zu versorgen. Der CAN-Transceiver funktioniert als Konverter zwischen der Physical-Layer-Schnittstelle und dem Mikrocontroller. Der Mikrocontroller ist für den Empfang der Dekodierung der vom CAN-Bus empfangenen Nachrichten und für die Steuerung der LED-Streifen verantwortlich.

%%本项目的硬件电路由四部分组成:电源模块,CAN转发器和微控制器。电源模块负责给整个硬件电路模块提供相应的电压。CAN转发器作为物理层接口和微控制器之间的转换器。微控制器负责接收从CAN总线%%上接收的消息的解码和LED灯带的控制。

1.1 Mikrocontroller
控制器是硬件电路的重要组成部分,也是LED灯带控制的核心。微控制器的主要任务是:
	其一,监听从CAN总线,接收CAN总线传来的指令
	其二:将接收来的指令进行解码
	其三:控制LED灯带显示相应的颜色。
在控制器的选择上,我们初步选择了两款控制芯片:STM32f103, Arduino due。这两款控制芯片都支持CAN总线。stm32芯片价格低廉,市场利用率高。arduino duo则是开源的芯片,开发简易,并且资源丰富,在这里选择了arduino duo 作为控制芯片。

1.2 CAN Transiver
CAN收发器将控制器连接到传输媒介。通常控制器和总线收发器通过光耦或磁耦隔离,这样即使总线上过压损坏收发器,控制器和Host设备也可以得到保护。

1.3 Spannungsquelle

1.4 LED Leiste
	Die LED Leiste mit LEDs vom Typ "WS2812b" wird in diese Projekt genutzt. Diese LED Leiste ist die Bauteile mit drei Draht. Und können mit unterschiedliche PWM die Farbe der LED steuern.



Die beiden Chips sind Highgeschwendigkeits CAN Transceiver, die zwischen CAN Kontroller und physikalische  Bus arbeiten können. Von dem Datenblatt kann man sehen, dass Aber Sie funktionieren unter unterschiedlichen Betriebsspannung. 


\end{document}